\documentclass[12pt, letterpaper]{article}
\usepackage[utf8]{inputenc}
\usepackage{amsfonts}
\usepackage{amsthm}

\title{Chapter 3 Notes}
\author{John J Li \thanks{notes from "How to Prove It" by Daniel J. Velleman}}
\date{\today}

\begin{document}
    \pagenumbering{gobble}

    \begin{titlepage}
        \maketitle
    \end{titlepage}

    \pagenumbering{arabic}

    \tableofcontents{}

    \newpage

    \section{Class Notes}

    Proof by contradiction:

    The aim is to show that a statement $r$ is true. The method is to show that an implication
    $\neg r \rightarrow F$ is true for some statement F which is false. 

    \subsection{Example}

    Let A,B,C be sets and $A \setminus B \in C$. Show that if $x \in A \setminus C$
    then $x \in B$. 

    \begin{proof}
        By contradiction: Suppose $x \in A \setminus C$ and $x \notin B$, then $x \in A$ and 
        $x \notin C$ and $x \notin B$. Thus $x \in A$ and $x \notin B$ and $x \notin C$. 
        Therefore $x \in A \setminus B$ and $x \notin C$. Since $A \setminus B \in C$, $x \in C$
        and $x \notin C$, which is a contradiction.
    \end{proof}

    \subsection{Example}

    Let $x,y \in \mathbb{R}$. Show that if $x^2 + y = 13$ and $y \neq 4$ then $x \neq 3$

    \begin{proof}
        Indirectly. 
        
        Suppose $x = 3$, we will show that $x^2 + y \neq 13$ or $y = 4$. 
        Suppose $x^2 + y = 13$. Then $3^2 + y = 13$, $y = 4$.

        Contradiction. 

        Suppose $x^2 + y = 13$ and $y \neq 4$ and $x = 3$. Then $9 + y = 13$ and $y\neq 4$.
        Therefore $y=4$ and $y\neq 4$, which is a contradiction. 
    \end{proof}

    \subsection{Example}

    Let $n \in \mathbb{Z}$. Show that if n is even then $n^2$ is even.

    \begin{proof}
        Suppose n is even. Then $n = 2k$ for some integer $k$. Then $n^2 = 4k^2 = 2(2k^2)$
        and $2k^2 \in\mathbb{Z}$. Therefore $n^2$ is even.
    \end{proof}

    \subsubsection{Additional}

    Show that if $n^2$ is even then n is even.

    \begin{proof}
        Suppose n is not even. Since $n\in \mathbb{Z}$, n is odd. 

        Then $n = 2k+1$ for some $k \in \mathbb{Z}$ and $n^2 = (2k+1)^2 = 4k^2 + 4k +1
        = 2(2k^2 + 2k) + 1$ and $2k^2 + 2k \in \mathbb{Z}$ Thus $n^2$ is odd.
    \end{proof}

    \subsection{Example}

    $\sqrt{2}$ is irrational

    \begin{proof}
        By contradiction. Suppose $\sqrt{2}$ is rational. Then $\sqrt{2} = \frac{p}{q}$ for
        some integers p,q where $q \neq 0$ and $\frac{p}{q}$ is reduced (p,q have no
        common factors).
        
        Then $\sqrt{2} \dot q = p$ and so $2q^2 = p^2$ (*). From (*) $p^2$ is even. Then, by
        the previous fact, p is even. Thus $p = 2k$ for some integer k.
        
        Thus in (*) we have $2q^2 = (2k)^2$; $2q^2 = 4k^2$; $q^2=2k^2$.
        Therefore $q^2$ is even, and so q is even.
        And so p,q are both even and $\frac{p}{q}$ is reduced which is a contradiction.
    \end{proof}

    \subsection{Example}

    Let A,B be sets. Show that if $A \land B = A$ then $A \in B$ ($A \in B \equiv \forall x
    \, x \in A, x \in B)$

    \begin{proof}
        Let $x \in A$. Since $A = A \cap B, x\in A \cap B$. Thus $x \in A$ and $x \in B$.
        In particular $x \in B$.
    \end{proof}

    \subsection{Example}

    Let $x \in \mathbb{R}$. Show that if $x>0$ then there is $y\in \mathbb{R}$ $y(y+1) = x$.
     $\forall x (x > 0 \rightarrow \exists y y (y+1) = x)$

    \begin{proof}
        Let $x>0$. Consider $ y = \frac{-1+\sqrt{1+4x}}{2}$ Then $ y \in \mathbb{R}$ because 
        $x > 0$ and $y(y+1) = (\frac{-1+\sqrt{1+4x}}{2})(\frac{-1+\sqrt{1+4x}}{2} + 1)$
        $=$
    \end{proof}


\end{document}