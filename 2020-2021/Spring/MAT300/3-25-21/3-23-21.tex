\documentclass{article}

\usepackage[a4paper, total={6in, 8in}]{geometry}
\usepackage[utf8]{inputenc}
\usepackage{amsmath}
\usepackage{amssymb}
\usepackage{amsthm}
\usepackage{fancyhdr}
\usepackage{enumitem}

\pagestyle{fancy}
\fancyhf{}
\lhead{John J Li}
\rhead{MAT300 Spring Notes}
\rfoot{\thepage}
\lfoot{March 2021}
\renewcommand{\headrulewidth}{0.4pt}
\renewcommand{\footrulewidth}{0.4pt}

\setlength{\parskip}{1em}
\setlength\parindent{0px}
\title{MAT300 Spring Notes}
\date{\today}
\author{John J Li}

\begin{document}
    \maketitle
    \thispagestyle{empty}
    \noindent\rule{\textwidth}{0.8pt}

    \section*{Notes}

    Order relations.

    \textbf{Definition 6} Let R be a relation on A. Then R is called a partial order on A if 
    it is reflexive, antisymmetric, and transitive.

    Examples: 1) $\leq$ on $\mathbb{R}$

    2) $\subseteq$ on $\mathcal{P}()\mathbb{N})$

    3) | on $\mathbb{Z}^+$

    A relation $R$ on $A$ is called a total order if it is a partial order and for every
    $a,b,\in A$ either $aRb$ or $bRa$
    
    Let $R$ be a partial order on $A$ and let $B$ be a subset of $A$.

    An element $b\in B$ is called minimal if there is no $b'$ in B such that $b'\neq b$
    and $b'Rb$.

    An element $b\in B$ is called maximal if there is no $b'$ in $B$ such that $b'\neq b$
    and $bRb'$.

    An element $b\in B$ is called smallest 



  


    %###################################################################################

    \section*{Example}

    a) $Z,\leq$

    $B = {-1,0,1,2,17}$

    minimal: $-1$

    smallest: $-1$
   
    b) $Z^+, |$

    $B = {2,3,5,6,15}$

    minimal: $2,3,5$

    smallest: none



\end{document}