\documentclass{article}

\usepackage[a4paper, total={6in, 8in}]{geometry}
\usepackage[utf8]{inputenc}
\usepackage{amsmath}
\usepackage{amssymb}
\usepackage{amsthm}
\usepackage{fancyhdr}
\usepackage{enumitem}

\pagestyle{fancy}
\fancyhf{}
\lhead{John J Li}
\rhead{MAT300 Spring Notes}
\rfoot{\thepage}
\lfoot{March 2021}
\renewcommand{\headrulewidth}{0.4pt}
\renewcommand{\footrulewidth}{0.4pt}

\setlength{\parskip}{1em}
\setlength\parindent{0px}
\title{MAT300 Spring Notes}
\date{\today}
\author{John J Li}

\begin{document}
    \maketitle
    \thispagestyle{empty}
    \noindent\rule{\textwidth}{0.8pt}

    \section*{Notes}

    The domain of R is:
    \[Dom(R)=\{a\in A|(a,b) \in R\}\]

    \textbf{Definition 5} Let $R$ be a relation from $A$ to $B$ and let $S$ be a relation from
    $B$ to $C$. Then the composition of $S$ and $R$ is the relation $S\dot R$ from $A$ to $C$.

    \[S\dot R=\{(a,c)\in A\times C|\exists_{b\in B}(a,b)\in R \land (b,c)\in S\}\].

    \textbf{Properties of relations}

    Let $R$ be a relation on $A$. ($R\in A\times A$)

    $R$ is called reflexive if for every $a\in A$. 
    \[aRa\]
    \[\forall_{a\in A} (a,a)\in R\]

    $R$ is called irreflexive if for every $a\in A$, 
    \[\neg aRa\]
    \[\forall_{a\in A} (a,a)\notin R\]

    $R$ is called symmetric if for every $a,b\in A$
    \[aRb\rightarrow bRa\]
    \[\forall_a\forall_b(a,b)\in R\rightarrow (b,a)\in R\]

    $R$ is called asymmetric if for every $a,b,\in A$,
    \[aRb\rightarrow\neg bRa\]
    \[\forall_a\forall_b(a,b)\in R\rightarrow (b,a)\notin R\]
    
    $R$ is antisymmetric if for every $a,b\in A$,
    \[(aRb\land bRa)\rightarrow(a=b)\] 
    \[\forall{a\in A}\forall_{b\in A}((a,b)\in R\land (b,a))\]

    $R$ is called transitive if for every $a,b,c\in A$,
    \[(aRb\land bRc)\rightarrow aRc\]


    %###################################################################################

    \section*{Examples}

    The relation of less than or equal to, $\leq$, on $R$.

    \begin{enumerate}
        \item reflexive: $\forall_{a\in\mathbb{R}} a=a$
        \item symmetric: $\forall_{a,b\in\mathbb{R}} a=b\rightarrow b=a$
        \item antisymmetric: $\forall_{a,b\in\mathbb{R}} (a=b\land b=a)\rightarrow a=b$
        \item transitive
    \end{enumerate}


    The relation of equality, $=$, on $R$

    

\end{document}