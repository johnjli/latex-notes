\documentclass{article}

\usepackage[a4paper, total={6in, 8in}]{geometry}
\usepackage[utf8]{inputenc}
\usepackage{fancyhdr}
\usepackage{graphicx}
\usepackage{enumitem}

\pagestyle{fancy}
\fancyhf{}
\lhead{John J Li}
\rhead{CSE360 Summer 2021 Notes}
\rfoot{\thepage}
\renewcommand{\headrulewidth}{0.4pt}

\setlength{\parskip}{1em}
\setlength\parindent{0px}
\title{CSE360 Summer 2021 Notes}
\date{\today}
\author{John J Li}

\begin{document}
    \section*{Information Management}

    Information Management System (IMS)
    \begin{itemize}
        \item General term for software systems designed to facilitate the storage, organization, and retrieval of information
        \item Sometimes used synonymously with Database Management System (DBMS)
        \item Data is stored in a variety of formats
        \item Makes it easier to access stored information
    \end{itemize}

    Steps when designing a database
    \begin{itemize}
        \item Requirements collection and analysis
        \item Conceptual design
        \item Choose a DBMS
        \item Logical Design
        \item Physical Design
        \item Implementation
    \end{itemize}

    \subsection*{IMS/DBSMS vs RDBMS}

    Information Management System (IMS)
    
    Database Management System (DBMS)
    \begin{itemize}
        \item Data is generally stored in either a hierachical or navigational form 
    \end{itemize}

    Relationship Database Management System (RDBMS)
    \begin{itemize}
        \item Data is stored in the form of tables.
        \item Is a special type of DBMS, which is based on a relational model
    \end{itemize}

    \subsection*{Data Model}

    Logical structure of a database -- describes the design of the database to 
    reflect entities, attributes, relationship among data, constraints, etc. 

    Types of data models:
    \begin{itemize}
        \item Object-based logical models - describes data at the conceptual 
        and view levels (etc. ER Model)
        \item Record-based logical models - specify logical structure of database 
        with records, fields, and attributes (etc. Relational Model)
    \end{itemize}

\end{document}