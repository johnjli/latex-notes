\documentclass{article}

\usepackage[a4paper, total={6in, 8in}]{geometry}
\usepackage[utf8]{inputenc}
\usepackage{fancyhdr}
\usepackage{graphicx}
\usepackage{enumitem}
\usepackage{listings}
\usepackage{color}

\renewcommand{\headrulewidth}{0.4pt}

\newcommand{\titleText}{CSE310 Lecture 1 Notes}

\pagestyle{fancy}
\fancyhf{}
\lhead{John J Li}
\rhead{\titleText}
\rfoot{\thepage}

\setlength{\parskip}{1em}
\setlength\parindent{0px}
\title{\titleText}
\date{\today}
\author{John J Li}

\definecolor{dkgreen}{rgb}{0,0.6,0}
\definecolor{gray}{rgb}{0.5,0.5,0.5}
\definecolor{mauve}{rgb}{0.58,0,0.82}

\lstset{frame=tb,
    language=Java,
    aboveskip=3mm,
    belowskip=3mm,
    showstringspaces=false,
    columns=flexible,
    basicstyle={\small\ttfamily},
    numbers=none,
    numberstyle=\tiny\color{gray},
    keywordstyle=\color{blue},
    commentstyle=\color{dkgreen},
    stringstyle=\color{mauve},
    breakatwhitespace=true,
    tabsize=3
}

\begin{document}
    \maketitle
    \thispagestyle{empty}
    \noindent\rule{\textwidth}{0.8pt}

    \section*{The Sorting Problem - Formal definition}

    Sorting problem:
    \begin{itemize}
        \item 
        Input: a sequence of $n$ numbers $<a_1,a_2,...,a_n>$.
        \item
        Output: a permetation (reordering) [of the input] $<a_1',a_2',...,a_n'>$
        such that $a_1'\leq a_2' \leq a_n'$
    \end{itemize}

    \subsection*{Insertion Sort: Analogy to storing cards...}

    There may be many ways to sort cards, but one intuitive procedure is to 
    always insert a card into the right place.
    \begin{itemize}
        \item Works naturally when you draw a stack of cards one-by-one and put them into 
        your hand.
        \item Similarily, if you sort all cards in your hand $\rightarrow$ \emph{"in place"} 
        sorting
    \end{itemize}

    \subsection*{Computationally represent the procedure}

    Pseudocode: Liberal user of English; Use of indentation for block structure; Omission
    of error handling (e.g. check if an array is empty).

    \begin{lstlisting}
        // Pseudocode for insertion sort 

        INSERTION-SORT(A)
        for j=2 to A.length // A.length=n
            key = A[j]
            // Insert A[j] into sorted sequence A[1...j-1]
            i=j-1
            while i>0 and A[i]>key
                A[i+1] = A[i]
                i=i-1
            A[i+1]=key
    \end{lstlisting}

    Example:
    \begin{center}
        \begin{tabular}{ccccc|l}
            3 & 9 & 6 & 1 & 2 & 9 should be inserted after 3 - no change \\
            3 & 9 & 6 & 1 & 2 & 6 should be inserted between 3 and 9 \\
            3 & 6 & 9 & 1 & 2 & 1 should be inserted before 3 \\
            1 & 3 & 6 & 9 & 2 & 2 should be inserted between 1 and 3 \\
            1 & 2 & 3 & 6 & 9 & sorted array\\
        \end{tabular}
    \end{center}

    \subsubsection*{Analysis of the Insertion-sort algorithm}

    When we analyze an algorithm, we mainly focus on the performance (running-time). 
    The running time of the algorithm depends on vaiour factors' success.
    \begin{itemize}
        \item depends on the nature of the input (already sorted vs. reverse sorted)
        \item depends on the size(n) ($6$ vs $10^6$ elements) $\rightarrow$ parameterized 
        running time.
        \begin{itemize}
            \item $T(n)=$ $\leftarrow$ function of the size (n).
        \end{itemize}
    \end{itemize}
    Different kinds of running time analysis:
    \begin{itemize}
        \item Worst-case.
        \begin{itemize}
            \item Max time on the input size $n$ given an upperbound guarentee to the user.
        \end{itemize}
        \item Average case.
        \begin{itemize}
            \item expected time over all possible input 
            \begin{itemize}
                \item running time of every input and find the average. We need to know
                the probability of each input $\rightarrow$ Input $\times$
                probability of that input = weighted average.
            \end{itemize}
            we don't know the exact probability, therefore, we make assumptions
            \begin{itemize}
                \item all inputs are equally likely -- uniform distibution.
            \end{itemize}
        \end{itemize}
        \item Best case
        \begin{itemize}
            \item works on some inputs only, no guarentee to the user.
        \end{itemize}
    \end{itemize}

    Next question: how to eliminate hardware dependency from running time.
    \begin{itemize}
        \item parametierize the running time based on the input size and then give the 
        growth of the running time rather than giving absolute value. $\rightarrow$ 
        asymptotic analysis (notation)
    \end{itemize}

    Asymptotic Notation:
    \begin{itemize}
        \item ignore machine dependent constants
        \item observe the growth of the running time.
    \end{itemize}
    One of the commonly used notation in the asymptotic noation is the $\theta$ notation.

    $\theta$ notation:
    \begin{itemize}
        \item drop lower order terms and ignore constants 
        \begin{itemize}
            \item $T(n) = 3n^3+50n^2+n+600$
            \begin{itemize}
                \item drop $50n^2+n+600$
                \item ignore $3$ in front of $3n^3$
            \end{itemize}
            \item $T(n)=\theta (n^3)$
        \end{itemize}
        \item this means when $n\rightarrow\infty$ $\theta(n^3)>\theta(n^2)$
    \end{itemize}

\end{document}